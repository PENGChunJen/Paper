\documentclass[sigconf]{acmart}

\usepackage{booktabs} % For formal tables
\usepackage[vlined, ruled]{algorithm2e}
\usepackage{multirow}


% Copyright
%\setcopyright{none}
%\setcopyright{acmcopyright}
\setcopyright{acmlicensed}
%\setcopyright{rightsretained}
%\setcopyright{usgov}
%\setcopyright{usgovmixed}
%\setcopyright{cagov}
%\setcopyright{cagovmixed}


% DOI
\acmDOI{10.475/123_4}

% ISBN
\acmISBN{123-4567-24-567/08/06}

%Conference
\acmConference[GECCO '17]{the Genetic and Evolutionary Computation Conference 2017}{July 15--19, 2017}{Berlin, Germany}
\acmYear{2017}
\copyrightyear{2017}


\begin{document}
\title{Two-edge Graphical Linkage Model for DSMGA-II}



\author{Ping-Lin Chen}
\authornote{These authors contributed equally to this work.}
\affiliation{%
  \institution{Taiwan Evolutionary Intelligence Laboratory}
  \institution{Department of Electrical Engineering}
  \institution{National Taiwan University}
}
\email{r04921043@ntu.edu.tw}

\author{Chun-Jen Peng}%\raise0.5ex\hbox{\small{*}}}
\authornote{These authors contributed equally to this work.}
\affiliation{%
  \institution{Taiwan Evolutionary Intelligence Laboratory}
  \institution{Department of Electrical Engineering}
  \institution{National Taiwan University}
}
\email{r04921039@ntu.edu.tw}

\author{Chang-Yi Lu}
\affiliation{%
  \institution{Taiwan Evolutionary Intelligence Laboratory}
  \institution{Department of Electrical Engineering}
  \institution{National Taiwan University}
}
\email{r03921053@ntu.edu.tw}

\author{Tian-Li Yu}
\affiliation{%
  \institution{Taiwan Evolutionary Intelligence Laboratory}
  \institution{Department of Electrical Engineering}
  \institution{National Taiwan University}
}
\email{tianliyu@ntu.edu.tw}



% The default list of authors is too long for headers}
\renewcommand{\shortauthors}{Chen et. al.}


\begin{abstract}

DSMGA-II, a model-based genetic algorithm, is capable of solving optimization problems via exploiting sub-structures of the problem. 
In terms of number of function evaluations (NFE), DSMGA-II has shown superior optimization ability to LT-GOMEA and hBOA on various benchmark problems as well as real-world problems. 
This paper proposes a two-edge graphical linkage model, which customizes recombination masks for each receiver according to its alleles, to further improve the performance of DSMGA-II.  
The new linkage model is more expressive than the original dependency structure matrix (DSM), providing far more possible linkage combinations than the number of solutions in the search space. 
To reduce unnecessary function evaluations, the two-edge model is used along with the supply bounds from the original DSM. 
%Some new techniques are also proposed to enhance the model selection efficiency. 
%Combining these proposed techniques, the empirical results show an average of 13\% NFE reduction on eight benchmark problems compared with the original DSMGA-II.
The empirical results show an average of 13\% NFE reduction on eight benchmark problems compared with the original DSMGA-II.
\end{abstract}

%
% The code below should be generated by the tool at
% http://dl.acm.org/ccs.cfm
% Please copy and paste the code instead of the example below. 
%
\begin{CCSXML}
<ccs2012>
<concept>
<concept_id>10010293.10010300.10010302</concept_id>
<concept_desc>Machine learning approaches~Maximum entropy modeling</concept_desc>
<concept_significance>500</concept_significance>
</concept>
<concept>
<concept_id>10010293.10011809.10011812</concept_id>
<concept_desc>Machine learning approaches~Genetic algorithms</concept_desc>
<concept_significance>500</concept_significance>
</concept>
<concept>
<concept_id>10010293.10010300.10010304</concept_id>
<concept_desc>Machine learning approaches~Mixture models</concept_desc>
<concept_significance>300</concept_significance>
</concept>	
</ccs2012> 
\end{CCSXML}

\ccsdesc[500]{Machine learning approaches~Maximum entropy modeling}
\ccsdesc[500]{Machine learning approaches~Genetic algorithms}
\ccsdesc[300]{Machine learning approaches~Mixture models}

% We no longer use \terms command
%\terms{Theory}

\keywords{Genetic Algorithm; Estimation-of-Distribution Algorithm; Linkage Learning; Model Building}


\maketitle

\input{samplebody-conf}

%\bibliographystyle{ACM-Reference-Format}
\bibliographystyle{abbrv}
\bibliography{sigproc} 

\end{document}
